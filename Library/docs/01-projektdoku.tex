\documentclass[10pt, a4paper]{scrartcl}
\usepackage{a4wide}
\usepackage[dvips]{graphicx}
\usepackage[utf8]{inputenc}
\usepackage{ngerman}

\title{Miniprojekt Bibliotheksverwaltung \\ \normalsize{Projektdokumentation}}
\author{Thomas Kallenberg, Martin Schwab}

\begin{document}
\maketitle
\section{Codequalität}
\subsection{Repository}
Das Repository wird von einem Hudsonserver überwacht.

\subsection{Coding-Styleguide}
Vor jedem Commit wird die Eclipse-Standardformatierung angewendet. Zudem werden die Empfehlungen von Sun (http://java.sun.com/docs/codeconv/html/CodeConvTOC.doc.html) für die Codeformatierung benutzt.

\section{Papier-Prototypen}
Die eingescannten Prototypen befinden sich aus Platzgründen in separaten Dateien im selben Verzeichnis wie dieses Dokument.

\subsection{Prototyp 1} Pro Aufgabe (Ausleihe, Rückgabe, Verwaltung, Reservation) steht hier ein Button zur Verfügung. Aber: Weiss man schon im Voraus, ob man ein Buch ausleihen oder reservieren will? Muss man es nicht zuerst suchen? Was bedeutet ''Verwaltung'' bzw. was wird verwaltet? Durch die geöffneten Fenster kann das Interface schnell unübersichtlich werden.

\subsection{Prototyp 2} Mit dem minimalistischen GUI konnten die Muss-Kriterien nicht erfüllt werden. Die Undo-History auf der rechten Seite lässt sich nur schwer realisieren. Da die Aktionen voneinander abhängig sein können, würde der Aufwand für die Implementation zu komplex. Das Menü links deckt zudem nicht alle Muss-Kriterien ab.

\subsection{Prototyp 3} Hier stellt sich beim Testen heraus, dass der Text auf dem Button ''Ausleihen'' irreführend ist, da er mit dem Verb ''ausleihen'' verwechselt wird. Der Knopf sollte alle ausgeliehenen Bücher anzeigen und dient nicht dazu, ein neues Buch auszuleihen. Ein weiterer Punkt sind die farbigen Bestätigungen, welche als lästig empfunden werden. Verbesserungsvorschlag: Eingelesene Bücher werden beim einscannen automatisch ausgeliehen / zurückgegeben und man kann den Vorgang in der Liste rückgängig machen. Statt dem Bestätigungs-Knopf soll ein ''zurück-zum-Hauptfenster''-Knopf angezeigt werden. Als letzter Mangel sei das ''R'' erwähnt, bei welchem nicht gleich klar ist, dass es für ''Reserviert'' stehen sollte. 

Und noch etwas: Der Laserscanner funktioniert wie ein normales Eingabegerät. Wird ein Code mit dieser Methode eingelesen, sollte dieser nicht an das Ende des Suchtextes angehängt werden. Ist man im Ausleihemodus, sollte der Suchtext unverändert bleiben, wenn man einen Buchcode einscannt.


\end{document}

