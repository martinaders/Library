% Install the following packages: beamer, xcolor and pgm

\documentclass[]{beamer}
\mode<presentation>{
 \usetheme{Rochester}			% schlichtes, blaues theme
 \setbeamercovered{transparent}		% Schattieren des verborgenen Textes
}
\usepackage[T1]{fontenc}
\usepackage[utf8]{inputenc}		% Support german umlauts
\usepackage[ngerman]{babel}		% Print german date
%\usepackage{pgfpages}			% Ausdruck auf mehrere Seiten
%\pgfpagesuselayout{4 on 1}[a4paper,landscape,border shrink=5mm]
\usepackage{graphicx}			% Graphiken benutzen

\pgfdeclareimage[height=1.5cm]{logo}{../src/img/splash}	% Logo rechts unten anzeigen
\logo{\pgfuseimage{logo}}

\title{Miniprojekt UserInterfaces -- Schlusspräsentation}
\author{Thomas Kallenberg, Martin Schwab}
\institute{HSR Hochschule Rapperswil}
\date{ \today }

\begin{document}

\begin{frame}
  \titlepage
\end{frame}

\begin{frame}{Übersicht}
\begin{itemize}
\item Was wurde umgesetzt
\item Demo
\item Technische Hintergründe
\item Fazit
\end{itemize}
\end{frame}

\begin{frame}{Was wurde umgesetzt}
\begin{columns}[t]
  \begin{column}{5cm}
  \begin{itemize}
    \item Menubar
    \item Tabs
    \item Liste mit eigener Suche
    \item Statusbar
    \item somemore..
  \end{itemize}

  \end{column}
  \begin{column}{5cm}
  \end{column}
\end{columns}
\end{frame}

\begin{frame}{Demonstration}
\begin{columns}[t]
  \begin{column}{5cm}
  \end{column}
  \begin{column}{5cm}
  \end{column}
\end{columns}
\end{frame}

\begin{frame}{Technische Hintergünde - Suche}
\textbf{Suchfunktion}
\begin{itemize}
\item<1-> Jeder Tastenanschlag wird weitergeleitet
  \begin{itemize}
  \item<2-> ein neues Zeichen wird angefügt
      \begin{itemize}
	\item<3-> letztes Suchresultat speichern, auf dessen Basis Anzeige weiter einschränken
      \end{itemize}
    \item<4-> das letzte Zeichen wird gelöscht
      \begin{itemize}
	\item<5-> vorletztes gespeichertes Suchresultat wird an die Stelle des aktuellen Resultats gesetzt
      \end{itemize}
  \end{itemize}
\item<6-> Bei unerwartetem, ganze Anfrage mit Zwischenresultaten erneut aufbauen. Langsam aber selten
\item<7-> Bibliothek geändert, ganze Suchanfrage erneut aufbauen
\item<8-> Suche fertig, Liste per Observer informieren
\invisible<1-8>{ \item . }
\end{itemize}
\end{frame}

\begin{frame}{Fazit}
\begin{itemize}
 \item Viele Elemente sind schwer zu verbinden. Architekturewissen hat gefehlt
 \item Viel Zeit bleibt an kleinen aber wichtigen Details hängen
 \item Refactoring in GUI ist schwierig und heikel
 \item git und github.com are our friends
 \item feel free to edit this
\end{itemize}

\end{frame}


\end{document}
